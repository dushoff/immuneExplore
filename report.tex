\documentclass[12pt]{article}

\bibliographystyle{plain}
\newcommand{\head}{\subsection*}
\newcommand{\sub}{\subsubsection*}

\begin{document}

\section*{Immune Exploration Final Report}

PHAC Contract 4500461000: \emph{Exploration of Immunity Patterns Using Dynamical
Model Simulation to Support Respiratory Disease
Modelling}. Submitted 21 Mar 2024

\head{Immunity gaps}

The COVID-19 pandemic caused severe disruptions to normal patterns of both public-health outreach and seasonal circulation of respiratory viruses. This raises the question of what to expect from these viruses, particularly influenza and RSV, in the coming years. There are several possible mechanisms by which the pandemic might have affected the immune landscape on which other viruses operate.

\sub{Generalized infection debt}

COVID lockdowns, particularly prior to omicron, led to lower \emph{overall} levels of infection in most populations. This caused concern about immune systems becoming “weakened”. Some of these concerns were naive, but there are also realistic reasons for concern, including cross-reactive T cells, Th1/Th2 balance, and innate immune mechanisms. Vaccination can stimulate the immune system is ways similar to infection, so lower levels of vaccination (see below) woud enhance this concern.

\sub{Specific infection debt}

This is the idea that we lose specific immune protection when not challenged by certain viruses, particularly flu and RSV. It's obviously true to some extent, and is a promising area for quantitative exploration (how much was circulation suppressed, how does serology look, what would simple models predict from a bounce-back).

\sub{Vaccination debt}

Vaccination programs have been directly affected by COVID disruptions, and indirectly affected by COVID-centered misinformation. To some extent this is continuing. Like ID2, this is obviously a serious concern and subject to quantitative exploration

\sub{Immune damage}

I don't understand the details yet, but it definitely seems plausible [see measles discoveries from a few years ago] that COVID infection (or maybe just moderate-to-severe COVID disease?) damages the immune system or immune memory.

\sub{Non-immune damage}

Similarly to immune damage, not well understood by me, but seems plausible. Lungs can have lingering damage and there could be synergies, e.g. [Pulmonary Dysfunction after Pediatric COVID-19](https://pubs.rsna.org/doi/10.1148/radiol.221250) There may be a role for secondary bacteria in this story [just made that up now]. 

\head{Partial immunity}

\bigskip\noindent Acquired immunity plays a critical role in the dynamics of infectious disease outbreaks \cite{anderson1985vaccination}. Individuals who have been exposed to a focal pathogen, or to a related pathogen, or who have been vaccinated, may be completely, or partially, immune.

Partial immunity is particularly challenging to understand, and to incorporate into models or forecasts. Two salient challenges for modeling are: defining outcomes, and population-level assumptions. The issues are slightly different when partial immunity is due to the focal pathogen (which may or may not be evolving), related pathogens, or inactivated vaccines. Live attenuated vaccines will not be considered in this project.

Measures of partial protection often differ based on the outcome measured. Modelers need to consider how much protection a population has against: infection (specifically, ability to transmit); clinical or reportable illnesss; and severe outcomes. The immune system has evolved primarily to protect \emph{individuals}; it is not surprising therefore that the effectiveness of partial protection generally increases as we move along this list [GodoyPMC6208006]. Assumptions about partial protection also have consequences for pathogen evolution \cite{gandon2003imperfect}.

Modelers attempting to model partial protection also need to consider how much partial immunity may “protect” individuals against “immune boosting” -- that is, developing further immunity.

Observations of partial protection are also subject to assumptions about population heterogeneity. Broadly, if a group of vaccinees (or recovered individuals) is observed to have 70\% protection against some outcome, this could mean that each individual is 70\% protected, or that 70\% of individuals are completely protected (with the rest not protected at all). The truth is almost certainly somewhere in between, but the majority of modeling approaches make one assumption or the other. These assumptions can be quite consequential.

These competing assumptions were outlined by Smith \cite{smith1984assessment} in the context of vaccination, and further developed by Halloran et al \cite{ halloran1991direct,halloran1992interpretation}, who also emphasized the importance for outcomes: “leaky” vaccines (which give each individual partial protection) would be associated with far larger outbreaks than “polarized” vaccines (where individual response is heterogeneous). Gog et al.~\cite{gog2002dynamics,gog2002status} constructed parallel ideas for competing strains, using the framework of “history-based” (keeping track of past exposures, analogous to the leaky framework) and “status-based” (projecting responses to future exposures, analogous to the polarized framework). In addition to short-term forecasting, these assumptions can have important, and sometimes surprisingly sharp, effects on long-term equilibria and responses to intervention measures and other parameter changes \cite{gomes2014missing}.

\bibliography{park}

\end{document}

